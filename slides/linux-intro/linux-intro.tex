\section{Introduction to Embedded Linux}

\subsection{Embedded Linux system architecture}

\begin{frame}
  \frametitle{Host and target}
  \begin{center}
    \includegraphics[height=0.8\textheight]{slides/linux-intro/host-and-target.pdf}
  \end{center}
\end{frame}

\begin{frame}
  \frametitle{Software components}
  \begin{itemize}
  \item Cross-compilation toolchain
    \begin{itemize}
    \item Compiler that runs on the development machine, but generates
      code for the target
    \end{itemize}
  \item Bootloader
    \begin{itemize}
    \item Started by the hardware, responsible for basic
      initialization, loading and executing the kernel
    \end{itemize}
  \item Linux Kernel
    \begin{itemize}
    \item Contains the process and memory management, network stack,
      device drivers and provides services to user space applications
    \end{itemize}
  \item C library
    \begin{itemize}
    \item Of course, a library of C functions
    \item Also the interface between the kernel and the user space
      applications
    \end{itemize}
  \item Libraries and applications
    \begin{itemize}
    \item Third-party or in-house
    \end{itemize}
  \end{itemize}
\end{frame}

\begin{frame}
  \frametitle{Embedded Linux work}

  Several distinct tasks are needed when deploying embedded Linux in a
  product:

  \begin{itemize}
  \item {\bf Board Support Package development}
    \begin{itemize}
    \item A BSP contains a bootloader and kernel with the suitable
      device drivers for the targeted hardware
    \end{itemize}
  \item {\bf System integration}
    \begin{itemize}
    \item Integrate all the components, bootloader, kernel,
      third-party libraries and applications and in-house applications
      into a working system
    \end{itemize}
  \item {\bf Development of applications}
    \begin{itemize}
    \item Normal Linux applications, but using specifically chosen
      libraries
    \end{itemize}
  \end{itemize}
\end{frame}
