\subsection{User-space interfaces to drivers}

\begin{frame}{User-space interfaces for hardware devices}

  For a high-level perspective: three main interfaces to access
  hardware devices exposed by the Linux kernel

  \begin{itemize}
  \item Device nodes in \code{/dev}
  \item Entries in the {\em sysfs} filesystem
  \item Network sockets and related APIs
  \end{itemize}
\end{frame}

\begin{frame}{Devices in {\em /dev/}}
  \begin{itemize}
  \item One of the kernel important roles is to {\bf allow applications
      to access hardware devices}
  \item In the Linux kernel, most devices are presented to user space
    applications through two different abstractions
    \begin{itemize}
    \item {\bf Character} device
    \item {\bf Block} device
    \end{itemize}
  \item Internally, the kernel identifies each device by a triplet of
    information
    \begin{itemize}
    \item {\bf Type} (character or block)
    \item {\bf Major} (typically the category of device)
    \item {\bf Minor} (typically the identifier of the device)
    \end{itemize}
  \item See \kfile{Documentation/admin-guide/devices.txt} for the
    official list of reserved type/major/minor numbers.
  \end{itemize}
\end{frame}

\begin{frame}{Block vs. character devices}
  \begin{itemize}
  \item Block devices
    \begin{itemize}
    \item A device composed of fixed-sized blocks, that can be read
      and written to store data
    \item Used for hard disks, USB keys, SD cards, etc.
    \end{itemize}
  \item Character devices
    \begin{itemize}
    \item Originally, an infinite stream of bytes, with no beginning,
      no end, no size. The pure example: a serial port.
    \item Used for serial ports, terminals, but also sound cards,
      video acquisition devices, frame buffers
    \item Most of the devices that are not block devices are
      represented as character devices by the Linux kernel
    \end{itemize}
  \end{itemize}
\end{frame}

\begin{frame}{Devices: everything is a file}
  \begin{itemize}
  \item A very important UNIX design decision was to represent most
    {\em system objects} as files
  \item It allows applications to manipulate all {\em system objects} with
    the normal file API (\code{open}, \code{read}, \code{write},
    \code{close}, etc.)
  \item So, devices had to be represented as files to the applications
  \item This is done through a special artifact called a {\bf device
      file}
  \item It is a special type of file, that associates a file name
    visible to user space applications to the triplet {\em (type,
      major, minor)} that the kernel understands
  \item All {\em device files} are by convention stored in the
    \code{/dev} directory
  \end{itemize}
\end{frame}

\begin{frame}[fragile]{Device files examples}

Example of device files in a Linux system

\small
\begin{verbatim}
$ ls -l /dev/ttyS0 /dev/tty1 /dev/sda /dev/sda1 /dev/sda2 /dev/sdc1 /dev/zero
brw-rw---- 1 root disk    8,  0 2011-05-27 08:56 /dev/sda
brw-rw---- 1 root disk    8,  1 2011-05-27 08:56 /dev/sda1
brw-rw---- 1 root disk    8,  2 2011-05-27 08:56 /dev/sda2
brw-rw---- 1 root disk    8, 32 2011-05-27 08:56 /dev/sdc
crw------- 1 root root    4,  1 2011-05-27 08:57 /dev/tty1
crw-rw---- 1 root dialout 4, 64 2011-05-27 08:56 /dev/ttyS0
crw-rw-rw- 1 root root    1,  5 2011-05-27 08:56 /dev/zero
\end{verbatim}
\normalsize

Example C code that uses the usual file API to write data to a serial port

\small
\begin{minted}{c}
int fd;
fd = open("/dev/ttyS0", O_RDWR);
write(fd, "Hello", 5);
close(fd);
\end{minted}
\end{frame}

\begin{frame}{Creating device files}
  \begin{itemize}
    \item Before Linux 2.6.32, on basic Linux systems,
    the device files had to be created manually using the
    \code{mknod} command
    \begin{itemize}
    \item \code{mknod /dev/<device> [c|b] major minor}
    \item Needs root privileges
    \item Coherency between device files and devices handled by the
      kernel was left to the system developer
    \end{itemize}
  \item The \code{devtmpfs} virtual filesystem can be mounted on
    \code{/dev} $\rightarrow$ the kernel automatically creates/removes
    device files
    \begin{itemize}
    \item \kconfig{CONFIG_DEVTMPFS_MOUNT} $\rightarrow$ asks the
      kernel to mount {\em devtmpfs} automatically at boot time
      (except when booting on an initramfs).
    \end{itemize}
  \end{itemize}
\end{frame}

\begin{frame}{Better handling of device files: {\em udev} and {\em mdev}}
  \begin{itemize}
  \item {\em devtmpfs} is great, but its capabilities are limited, so
    complementary solutions exist
  \item {\bf udev}
    \begin{itemize}
    \item daemon that receives events from the kernel about devices
      appearing/disappearing
    \item can create/remove device files (but that's done by
      {\em devtmpfs} now), adjust permission/ownership,
      load kernel modules automatically, create symbolic links to
      devices
    \item according to rules files in \code{/lib/udev/rules.d} and
      \code{/etc/udev/rules.d}
    \item used in almost all desktop Linux distributions
    \item \url{https://en.wikipedia.org/wiki/Udev}
    \end{itemize}
  \item {\bf mdev}
    \begin{itemize}
    \item lightweight implementation of {\em udev}, part of Busybox
    \item \url{https://wiki.gentoo.org/wiki/Mdev}
    \end{itemize}
  \end{itemize}
\end{frame}

\begin{frame}{Examples of user-space interfaces in {\tt /dev}}
  \begin{itemize}
  \item Serial-ports: \code{/dev/ttyS*}, \code{/dev/ttyUSB*},
    \code{/dev/ttyACM*}, etc.
  \item GPIO controllers (modern interface): \code{/dev/gpiochipX}
  \item Block storage devices: \code{/dev/sd*}, \code{/dev/mmcblk*}, \code{/dev/nvme*}
  \item Flash storage devices: \code{/dev/mtd*}
  \item Display controllers and GPUs: \code{/dev/dri/*}
  \item Audio devices: \code{/dev/snd/*}
  \item Camera devices: \code{/dev/video*}
  \item Watchdog devices: \code{/dev/watchdog*}
  \item Input devices: \code{/dev/input/*}
  \item and many more...
  \end{itemize}
\end{frame}

\begin{frame}{{\em sysfs} filesystem}
  \begin{itemize}
  \item \code{block/}, symlinks to all block devices, in
    \code{/sys/devices}
  \item \code{bus/}, one sub-folder by type of bus
  \item \code{class/}, one sub-folder per class (category of devices):
    input, leds, pwm, etc.
  \item \code{dev/}
    \begin{itemize}
    \item \code{block/}, one symlink per block device, named after
      major/minor
    \item \code{char/}, one symlink per character device, named after
      major/minor
    \end{itemize}
  \item \code{devices/}, all devices in the system, organized in a
    slightly chaotic way, see \href{https://lwn.net/Articles/646617/}{this article}
  \item \code{firmware/}, representation of firmware data
    \begin{itemize}
    \item \code{devicetree/}, directory and file representation of
      Device Tree nodes and properties
    \end{itemize}
  \item \code{fs/}, properties related to filesystem drivers
  \item \code{kernel/}, properties related to various kernel subsystems
  \item \code{module/}, properties about kernel modules
  \item \code{power/}, power-management related properties
  \end{itemize}
\end{frame}

\begin{frame}[fragile]{{\em sysfs} filesystem example}
  \begin{itemize}
  \item \code{/sys/bus/i2c/drivers}: all device drivers for devices
    connected on I2C busses
    \begin{block}{}
      {\tiny
\begin{verbatim}
[...]
edt_ft5x06
stpmic1
[...]
\end{verbatim}
      }
    \end{block}

  \item \code{/sys/bus/i2c/devices}: all devices in the system
    connected to I2C busses
    \begin{block}{}
      {\tiny
\begin{verbatim}
0-002a -> ../../../devices/platform/soc/40012000.i2c/i2c-0/0-002a
0-0039 -> ../../../devices/platform/soc/40012000.i2c/i2c-0/0-0039
0-004a -> ../../../devices/platform/soc/40012000.i2c/i2c-0/0-004a
1-0028 -> ../../../devices/platform/soc/5c002000.i2c/i2c-1/1-0028
1-0033 -> ../../../devices/platform/soc/5c002000.i2c/i2c-1/1-0033
i2c-0 -> ../../../devices/platform/soc/40012000.i2c/i2c-0
i2c-1 -> ../../../devices/platform/soc/5c002000.i2c/i2c-1
i2c-2 -> ../../../devices/platform/soc/40012000.i2c/i2c-0/i2c-2
\end{verbatim}
      }
    \end{block}
  \end{itemize}
\end{frame}

\begin{frame}[fragile]{{\em sysfs} filesystem example}
  \begin{block}{/sys/bus/i2c/devices/0-002a/}
    {\tiny
\begin{verbatim}
lrwxrwxrwx    driver -> ../../../../../../bus/i2c/drivers/edt_ft5x06
-rw-r--r--    gain
drwxr-xr-x    input
-r--r--r--    modalias
-r--r--r--    name
lrwxrwxrwx    of_node -> ../../../../../../firmware/devicetree/base/soc/i2c@40012000/touchscreen@2a
-rw-r--r--    offset
-rw-r--r--    offset_x
-rw-r--r--    offset_y
drwxr-xr-x    power
-rw-r--r--    report_rate
lrwxrwxrwx    subsystem -> ../../../../../../bus/i2c
-rw-r--r--    threshold
-rw-r--r--    uevent
\end{verbatim}
    }
  \end{block}

  \begin{itemize}
  \item \code{driver}, symlink to the driver directory in \code{/sys/bus/i2c/drivers}
  \item \code{of_node}, symlink to the directory for the Device Tree
    node describing this device
  \end{itemize}
\end{frame}

\begin{frame}{Example of driver interfaces in {\em sysfs}}
  \begin{itemize}
  \item All devices are visible in {\em sysfs}, whether they have an
    interface in \code{/dev} or not
    \begin{itemize}
    \item Usually \code{/dev} is to access the device
    \item \code{/sys} is more about properties of the devices
    \end{itemize}
  \item However, some devices only have a {\em sysfs} interface
    \begin{itemize}
    \item LED: \code{/sys/class/leds}, see \href{https://docs.kernel.org/leds/leds-class.html}{documentation}
    \item PWM: \code{/sys/class/pwm}, see \href{https://docs.kernel.org/driver-api/pwm.html\#using-pwms-with-the-sysfs-interface}{documentation}
    \item IIO: \code{/sys/bus/iio}, see \href{https://docs.kernel.org/driver-api/iio/index.html}{documentation}
    \item etc.
    \end{itemize}
  \end{itemize}
\end{frame}

\begin{frame}{Accessing GPIOs}
  A class of devices worth mentioning is GPIOs ({\em General Purpose Input Output})
  \begin{itemize}
  \item The GPIOs can be accessed through a legacy interface in
        \code{/sys/class/gpios}
    \begin{itemize}
       \item You will find many instructions on the Internet about how
	     to drive GPIOs through this interface.
       \item However, this interface is deprecated and has multiple
             shortcomings:
             \begin{itemize}
		\item GPIOs remain exported if the process using them crashes
		\item Need to compute the GPIO numbers, such numbers are not stable
	     \end{itemize}
    \end{itemize}
  \item A new interface recommended: \href{https://git.kernel.org/pub/scm/libs/libgpiod/libgpiod.git/}{libgpiod}
    \begin{itemize}
	\item Based on \code{/dev/gpiochipx} character devices
        \item Implementing advanced features not possible with the legacy interface
	\item Of course, this is a C library
        \item But it also provides command line utilities:
              \code{gpiodetect}, \code{gpioset}, \code{gpioget}...
	\item The only constraint is to cross-compile them for your target
	      (the legacy interface could be used without any additional software).
    \end{itemize}
  \end{itemize}
\end{frame}

\begin{frame}{Other virtual filesystems}

  \begin{itemize}

  \item {\em debugfs}
    \begin{itemize}
    \item Conventionally mounted in \code{/sys/kernel/debug}
    \item Contains lots of debug information from the kernel,
      including device related
    \item \code{/sys/kernel/debug/pinctrl} for pin-mux debugging,
      \code{/sys/kernel/debug/gpio} for GPIO debugging,
      \code{/sys/kernel/debug/pwm} for PWM debugging, etc.
    \item \url{https://www.kernel.org/doc/html/latest/filesystems/debugfs.html}
    \end{itemize}
  \item {\em configfs}
    \begin{itemize}
    \item Conventionally mounted in \code{/sys/kernel/config}
    \item Allows to manage configuration of advanced kernel mechanisms
    \item Example: configuration of USB gadget functionalities
    \item \kfile{Documentation/filesystems/configfs.rst}
    \end{itemize}
  \end{itemize}
\end{frame}
