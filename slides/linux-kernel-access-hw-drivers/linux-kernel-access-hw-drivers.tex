\subsection{Kernel drivers}

\begin{frame}{Typical software stack for hardware access}
  \begin{columns}
    \column{0.6\textwidth}
    From the bottom to the top:
    \begin{itemize}
    \item A {\em bus controller driver} in the kernel drives an I2C,
      SPI, USB, PCI controller
    \item A {\em bus subsystem} provides an API for drivers to access
      a particular type of bus: I2C, SPI, PCI, USB, etc.
    \item A {\em device driver} in the kernel drives a particular
      device connected to a given bus
    \item A {\em driver subsystem} exposes features of certain class
      of devices, through a standard {\em kernel/user-space interface}
    \item An application can access the device through this standard {\em
        kernel/user-space interface} either directly or through a
      library.
    \end{itemize}
    \column{0.4\textwidth}
    \begin{center}
      \includegraphics[height=0.7\textheight]{slides/linux-kernel-access-hw-drivers/kernel-driver-stack.pdf}
    \end{center}
  \end{columns}
\end{frame}

\begin{frame}{Stack illustrated with a GPIO expander}
  \begin{center}
    \includegraphics[height=0.8\textheight]{slides/linux-kernel-access-hw-drivers/kernel-driver-stack-gpio-i2c.pdf}
  \end{center}
\end{frame}

\begin{frame}{Standardized user-space interface}
  \begin{itemize}
  \item Strong advantage of kernel drivers: they expose a standard
    {\em kernel to user-space interface}
  \item All devices of the same class (e.g GPIO controllers) will
    expose the same {\em kernel to user-space interface}
  \item Applications don't have to know the details of the GPIO
    controller, they just need to know the standard user-space
    interface valid for all GPIO controllers
  \item Applications can use existing open-source libraries that
    leverage this standard user-space interface
  \item Such kernel drivers can also be used internally inside the
    kernel, for example if one driver needs to control a GPIO directly
    (reset signal, interrupt signal, etc.)
  \end{itemize}
\end{frame}

\begin{frame}{Numerous kernel subsystems for device classes}
  \begin{columns}
    \column{0.5\textwidth}
    \begin{itemize}
    \item Networking stack for Ethernet, WiFi, CAN, 802.15.4, etc.
    \item GPIO
    \item Video4Linux for camera, video encoders/decoders
    \item DRM for display controllers, GPU
    \item ALSA for audio
    \item IIO for ADC, DAC, gyroscopes, sensors, and more
    \item MTD for flash memory
    \item PWM
    \end{itemize}
    \column{0.5\textwidth}
    \begin{itemize}
    \item Input for keyboard, mouse, touchscreen, joystick
    \item Watchdog
    \item RTC for real-time clocks
    \item remoteproc for auxiliary processors
    \item crypto for cryptographic accelerators
    \item hwmon for hardware monitoring sensors
    \item block layer for block storage
    \end{itemize}
  \end{columns}
  \begin{center}
    and many more
  \end{center}
\end{frame}

\begin{frame}{Accessing devices directly from user-space}
  \begin{itemize}
  \item Even though device drivers in the kernel are preferred, it is
    also possible to access devices directly from user-space
  \item Especially useful for very specific devices that do not fit in
    any existing kernel subsystems
  \item The kernel provides the following mechanisms, depending on the
    bus:
    \begin{itemize}
    \item I2C: \href{https://docs.kernel.org/i2c/dev-interface.html}{i2c-dev}
    \item SPI: \href{https://docs.kernel.org/spi/spidev.html}{spidev}
    \item Memory-mapped: \href{https://docs.kernel.org/driver-api/uio-howto.html}{UIO}
    \item USB: \code{/dev/bus/usb}, through \href{https://libusb.info/}{libusb}
    \item PCI: \href{https://docs.kernel.org/PCI/sysfs-pci.html}{sysfs entries for PCI}
    \end{itemize}
  \end{itemize}
\end{frame}

\begin{frame}{Accessing devices directly from user-space: GPIO example}
  \begin{columns}
    \column{0.3\textwidth} {\small This diagram shows what's not
      recommended to do $\rightarrow$ for a GPIO controller, a kernel driver
      is preferred}
    \column{0.7\textwidth}
    \begin{center}
      \includegraphics[height=0.8\textheight]{slides/linux-kernel-access-hw-drivers/kernel-driver-stack-gpio-i2c-direct-userspace.pdf}\\
    \end{center}
  \end{columns}
\end{frame}

\begin{frame}{What can go wrong with a user-space driver?}
  \begin{itemize}
  \item You write your GPIO driver in user-space: other kernel drivers
    cannot use GPIOs from this GPIO controller
    \begin{itemize}
    \item Other devices that use GPIO signals from this controller for reset, interrupt, etc. cannot control/configure those signals
    \item Your application is less portable: it will take many changes to support another type of GPIO controller.
    \end{itemize}
  \item You write your touchscreen driver in user-space: the standard
    Linux graphics stack components cannot use your touchscreen
  \item You write your network driver in user-space
    \begin{itemize}
    \item You can probably send/receive packets
    \item But you cannot leverage the Linux kernel networking stack
      for IP, TCP, UDP, etc.
    \item And none of the Linux networking applications can use your
      network device
    \end{itemize}
  \end{itemize}
\end{frame}

\begin{frame}{Upstream drivers vs. out-of-tree drivers}
  \begin{itemize}
  \item The {\em upstream} Linux kernel contains thousands of drivers
    \begin{itemize}
    \item This is the best place to look for drivers
    \item Drivers have been reviewed and approved by the community
    \item They comply with standard interfaces
    \end{itemize}
  \item Vendor kernels often include additional drivers, directly in
    the kernel tree
  \item Device vendors sometimes also provide {\em out of tree
      drivers}
    \begin{itemize}
    \item Their source code is provided separately from the Linux
      kernel tree
    \item Quality is often dubious
    \item Compatibility issues when updating to newer kernel releases
    \item Not always use standard user-space interfaces
    \item Example: \url{https://github.com/lwfinger/rtl8723ds}
    \item Avoid them when possible!
    \end{itemize}
  \end{itemize}
\end{frame}

\begin{frame}{Finding Linux kernel drivers}
  \begin{itemize}
  \item \code{grep} in the Linux kernel tree is your {\em best friend}
    \begin{itemize}
    \item For I2C, SPI and memory-mapped devices, matching of the
      driver is done based on the device name $\rightarrow$ {\em grep}
      for variants of the device name and vendor
    \item For USB, PCI, matching is done either on the vendor
      ID/product ID, or the class $\rightarrow$ {\em grep} for these
    \end{itemize}
  \item Driver file names are sometimes named in a ``generic'' way,
    not necessarily reflecting all devices they support.
    \begin{itemize}
    \item Example: \kfile{drivers/gpio/gpio-pca953x.c} supports much
      more than just PCA953x. See the
      \href{https://elixir.bootlin.com/linux/v5.19/source/drivers/gpio/gpio-pca953x.c\#L1221}{full
        list of devices} supported by this driver
    \end{itemize}
  \end{itemize}
\end{frame}

\begin{frame}[fragile]{Finding Linux kernel drivers: an example}
  \begin{itemize}
  \item You have a
    \href{https://www.maximintegrated.com/en/products/interface/controllers-expanders/MAX7313.html}{Maxim
      Integrated MAX7313} GPIO expander on I2C
  \item Search in the Linux kernel
    \begin{block}{git grep -i max7313}
      {\tiny
\begin{verbatim}
drivers/gpio/gpio-pca953x.c:    { "max7313", 16 | PCA953X_TYPE | PCA_INT, },
drivers/gpio/gpio-pca953x.c:    { .compatible = "maxim,max7313", .data = OF_953X(16, PCA_INT), },
\end{verbatim}
      }
    \end{block}
  \item \kfile{drivers/gpio/gpio-pca953x.c} seems to support it
  \item Read \kfile{drivers/gpio/Makefile} to learn which kernel
    configuration option enables this driver
    \begin{block}{\kfile{drivers/gpio/Makefile}}
      {\tiny
\begin{verbatim}
obj-$(CONFIG_GPIO_PCA953X)              += gpio-pca953x.o
\end{verbatim}
      }
    \end{block}
  \item Conclusion: you need to enable \kconfig{CONFIG_GPIO_PCA953X} in
    your kernel configuration
  \end{itemize}
\end{frame}
