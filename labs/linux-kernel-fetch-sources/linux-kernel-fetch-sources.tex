\subchapter{Fetching Linux kernel sources}{Objective: fetch
the Linux kernel sources from git, from both the master and stable
branches.}

\section{Cloning the mainline Linux tree}

To begin working with the Linux kernel sources, we need to clone its
reference git tree, the one managed by Linus Torvalds.

\begin{bashinput}
$ git clone https://git.kernel.org/pub/scm/linux/kernel/git/torvalds/linux
$ cd linux
\end{bashinput}

\section{Accessing stable releases}

The Linux kernel repository from Linus Torvalds contains all the main
releases of Linux, but not the stable versions: they are maintained by
a separate team, and hosted in a separate repository.

We will add this separate repository as another {\em remote} to be
able to use the stable releases:

\begin{bashinput}
$ git remote add stable https://git.kernel.org/pub/scm/linux/kernel/git/stable/linux
$ git fetch stable
\end{bashinput}

\section{Choose a particular stable version of Linux}

We will use the Linux \workingkernel,
the remote branch we are interested in is
\texttt{remotes/stable/linux-\workingkernel.y}.

First, execute the following command to check which version you
currently have:

\begin{bashinput}
$ make kernelversion
\end{bashinput}

You can also open the \code{Makefile} and look at the beginning of it
to check this information.

Now, let's create a local branch starting from that remote branch:

\begin{bashinput}
$ git checkout stable/linux-%\workingkernel%.y
\end{bashinput}

Check the version again using the \code{make kernelversion} command
to make sure you now have a \workingkernel.x version.

\begin{notebox}
Personal workspace: \url{https://github.com/ngocho31/linux.git}
\end{notebox}
