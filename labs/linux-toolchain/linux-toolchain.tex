\subchapter{Building a cross-compiling toolchain}{Objective: Compile
 a cross-compiling toolchain for the musl C library}

\section{Getting Crosstool-ng}

Download the sources of Crosstool-ng, through its git source repository,
and switch to a commit that we have tested:

\begin{bashinput}
$ git clone https://github.com/crosstool-ng/crosstool-ng
$ cd crosstool-ng/
$ git checkout crosstool-ng-1.27.0
\end{bashinput}

\section{Building and installing Crosstool-ng}

As we are not building Crosstool-ng from a release archive but from
a git repository, we first need to generate a \code{configure} script and
more generally all the generated files that are shipped in the source
archive for a release:

\begin{bashinput}
$ ./bootstrap
\end{bashinput}

We can then either install Crosstool-ng globally on the system, or keep it
locally in its download directory. We'll choose the latter
solution. As documented at
\url{https://crosstool-ng.github.io/docs/install/#hackers-way}, do:

\begin{bashinput}
$ ./configure --enable-local
$ make
\end{bashinput}

Then you can get Crosstool-ng help by running

\begin{bashinput}
$ ./ct-ng help
\end{bashinput}

\section{Configure the toolchain to produce}

A single installation of Crosstool-ng allows to produce as many
toolchains as you want, for different architectures, with different C
libraries and different versions of the various components.

Crosstool-ng comes with a set of ready-made configuration files for
various typical setups: Crosstool-ng calls them {\em samples}. They can be
listed by using \code{./ct-ng list-samples}.

We will load the
% \ifdefstring{\labboard}{qemu}{Cortex A9}{}%
\ifdefstring{\labboard}{beaglebone}{Cortex A8}{}
% \ifdefstring{\arch}{ARM64}{\code{aarch64-unknown-linux-uclibc}}{}%
% \ifdefstring{\labboard}{stm32mp1}{Cortex A5}{}
sample%
% \ifdefstring{\labboard}{stm32mp1}{, as Crosstool-ng doesn't have any sample for Cortex A7 yet}{}
. Load it with the \code{./ct-ng}
\ifdefstring{\labboard}{beaglebone}{\code{arm-cortex_a8-linux-gnueabi}}{}
command.

Then, to refine the configuration, let's run the \code{menuconfig} interface:

\begin{bashinput}
$ ./ct-ng menuconfig
\end{bashinput}

In \code{Path and misc options}:
\begin{itemize}
\item If not set yet, enable \code{Try features marked as EXPERIMENTAL}
\end{itemize}

% \ifdefstring{\labboard}{stm32mp1}{
% In \code{Target options}:
% \begin{itemize}
% \item Set \code{Emit assembly for CPU} (\code{ARCH_CPU}) to \code{cortex-a7}.
% \item Set \code{Use specific FPU} (\code{ARCH_FPU}) to \code{vfpv4}.
% \item Set \code{Floating point} to \code{hardware (FPU)}.
% \end{itemize}
% }{}

\ifdefstring{\labboard}{beaglebone}{
In \code{Target options}:
\begin{itemize}
\item Set \code{Use specific FPU} (\code{ARCH_FPU}) to \code{vfpv3}.
\item Set \code{Floating point} to \code{hardware (FPU)}.
\end{itemize}
}{}

% \ifdefstring{\arch}{ARM64}{
% In \code{Target options}:
% \begin{itemize}
% \item Set \code{Emit assembly for CPU} (\code{ARCH_CPU}) to \code{cortex-a53}.
% \item Check that \code{Endianness} (\code{ARCH_ENDIAN}) is set to \code{Little endian}
% \end{itemize}
% }{}

In \code{Toolchain options}:
\ifdefstring{\arch}{ARM64}{
\begin{itemize}
\item Set \code{Tuple's vendor string} (\code{TARGET_VENDOR}) to \code{none}.
\item Set \code{Tuple's alias} (\code{TARGET_ALIAS}) to \code{aarch64-linux}.
      This way, we will be able to use the compiler as \code{aarch64-linux-gcc}
      instead of \code{aarch64-training-linux-musl-gcc}, which is
      much longer to type.
\end{itemize}
}
{
\begin{itemize}
\item Set \code{Tuple's vendor string} (\code{TARGET_VENDOR}) to \code{none}.
\item Set \code{Tuple's alias} (\code{TARGET_ALIAS}) to
  \code{arm-linux}.  This way, we will be able to use the compiler
  with \code{arm-linux-gcc}, a shorter name that the name based on
  complete toolchain tuple.
\end{itemize}
}

In \code{Operating System}:
\begin{itemize}
\item Set \code{Version of linux} to the closest, but older version to
      \workingkernel. It's important that the kernel headers used in
      the toolchain are not more recent than the kernel that will run
      on the board (v\workingkernel).
\end{itemize}

In \code{C-library}:
\begin{itemize}
  \item If not set yet, set \code{C library} to \code{musl}
        (\code{LIBC_MUSL})
  \item Keep the default version that is proposed
\end{itemize}

In \code{C compiler}:
\begin{itemize}
  \item Set \code{Version of gcc} to \code{14.2.0}.
  \item Make sure that \code{C++} (\code{CC_LANG_CXX}) is enabled
\end{itemize}

In \code{Debug facilities}:
\begin{itemize}
\item Remove all options here. Some debugging tools can be provided
      in the toolchain, but they can also be built by filesystem
      building tools.
\end{itemize}

Explore the different other available options by traveling through the
menus and looking at the help for some of the options.

\section{Produce the toolchain}

Simple:

\begin{bashinput}
$ ./ct-ng build
\end{bashinput}

The toolchain will be installed by default in \code{$HOME/x-tools/}.
That's something you could have changed in Crosstool-ng's configuration.

\section{Testing the toolchain}

You can now test your toolchain by adding
\ifdefstring{\arch}{ARM64}{\code{$HOME/x-tools/aarch64-none-linux-musl/bin/}}
{\code{$HOME/x-tools/arm-none-linux-musleabihf/bin/}} to your
\code{PATH} environment variable:

\begin{bashinput}
$ export PATH=$PATH:$HOME/x-tools/%\ifdefstring{\arch}{ARM64}{aarch64-none-linux-musl}{arm-none-linux-musleabihf}%/bin/
\end{bashinput}

Compiling the simple \code{hello.c} program with
\ifdefstring{\arch}{ARM64}{\code{aarch64-linux-gcc}}{\code{arm-linux-gcc}}:

\begin{bashinput}
$ %\ifdefstring{\arch}{ARM64}{aarch64-linux-gcc}{arm-linux-gcc}% -o hello hello.c
\end{bashinput}

You can use the \code{file} command on your binary to make sure it has
correctly been compiled for the \ifdefstring{\arch}{ARM64}{AARCH64}{ARM}
architecture.

\if\defstring{\arch}{ARM64}
\else
  \begin{bashinput}
  $ file hello
  hello: ELF 32-bit LSB executable, ARM, EABI5 version 1 (SYSV), dynamically linked, interpreter /lib/ld-musl-armhf.so.1, not stripped
  \end{bashinput}
\fi

\subsection{Execute this binary from your x86 host}

Run QEMU \ifdefstring{\arch}{ARM64}{AARCH64}{ARM} user emulator:

\if\defstring{\arch}{ARM64}
  \begin{bashinput}
  $ qemu-aarch64 hello
  qemu-aarch64: Could not open '/lib/ld-musl-aarch64.so.1': No such file or directory
  \end{bashinput}
\else
  \begin{bashinput}
  $ qemu-arm hello
  qemu-arm: Could not open '/lib/ld-musl-armhf.so.1': No such file or directory
  \end{bashinput}
\fi

\ifdefstring{\arch}{ARM64}{\code{qemu-aarch64}}{\code{qemu-arm}}
is missing the shared library loader
(compiled for \ifdefstring{\arch}{ARM64}{AARCH64}{ARM}) that this
binary relies on.

Find it in our newly compiled toolchain:

\begin{bashinput}
$ find ~/x-tools -name %\ifdefstring{\arch}{ARM64}{ld-musl-aarch64.so.1}{ld-musl-armhf.so.1}%
\end{bashinput}

\if\defstring{\arch}{ARM64}
  \begin{terminaloutput}
  ~/x-tools/aarch64-none-linux-musl/aarch64-none-linux-musl/sysroot/lib/ld-musl-aarch64.so.1
  \end{terminaloutput}
\else
  \begin{terminaloutput}
  ~/x-tools/arm-none-linux-musleabihf/arm-none-linux-musleabihf/sysroot/lib/ld-musl-armhf.so.1
  \end{terminaloutput}
\fi

We can now use the \code{-L} option of \ifdefstring{\arch}{ARM64}
{\code{qemu-aarch64}}{\code{qemu-arm}} to let it know where shared libraries are:

\if\defstring{\arch}{ARM64}
  \begin{bashinput}
  $ qemu-aarch64 -L ~/x-tools/aarch64-none-linux-musl/aarch64-none-linux-musl/sysroot hello
  \end{bashinput}
\else
  \begin{bashinput}
  $ qemu-arm -L ~/x-tools/arm-none-linux-musleabihf/arm-none-linux-musleabihf/sysroot hello
  \end{bashinput}
\fi

\begin{terminaloutput}
Hello world!
\end{terminaloutput}

\section{Cleaning up}

{\em Do this only if you have limited storage space. In case you made a
mistake in the toolchain configuration, you may need to run Crosstool-ng
again, keeping generated files would save a significant amount of time.}

To save about 9 GB of storage space, do a \code{./ct-ng clean} in the
Crosstool-NG source directory. This will remove the source code of the
different toolchain components, as well as all the generated files
that are now useless since the toolchain has been installed in
\code{$HOME/x-tools}.

\section{Output}

\begin{itemize}
\item \href{https://github.com/ngocho31/crosstool-ng/tree/bbb_v1.26.0}{Crosstool 1.26 for BBB simple project}
\item \href{https://github.com/ngocho31/crosstool-ng/tree/bbb_v1.27.0}{Crosstool 1.27 for BBB simple project}
\end{itemize}
