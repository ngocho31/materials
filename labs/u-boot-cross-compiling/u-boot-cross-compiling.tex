\subchapter{Bootloader - U-Boot - Cross-compiling}{Objectives:
Compile the U-Boot bootloader}

As the bootloader is the first piece of software executed by a
hardware platform, the installation procedure of the bootloader is
very specific to the hardware platform. There are usually two cases:

\begin{itemize}

\item The processor offers nothing to ease the installation of the
  bootloader, in which case the JTAG has to be used to initialize
  flash storage and write the bootloader code to flash. Detailed
  knowledge of the hardware is of course required to perform these
  operations.

\item The processor offers a monitor, implemented in ROM, and through
  which access to the memories is made easier.

\end{itemize}

\ifdefstring{\labboard}{beaglebone}
{
The AM3358 SoC on the BeagleBone falls into the second category. The monitor
integrated in the ROM reads the SD card to search for a valid
bootloader.
}{}

\section{Compiling U-Boot and SPL}

Get an understanding of U-Boot's configuration and compilation steps
by reading the \code{README} file, and specifically the {\em Building
the Software} section.

Basically, you need to:

\begin{enumerate}
\item Specify the cross-compiler prefix
(the part before \code{gcc} in the cross-compiler executable name):

\begin{bashinput}
$ export PATH=$PATH:$HOME/x-tools/%\ifdefstring{\arch}{ARM64}{aarch64-none-linux-musl}{arm-none-linux-musleabihf}%/bin/
$ export CROSS_COMPILE=arm-linux-
\end{bashinput}

\ifdefstring{\labboard}{beaglebone}
{
\item Run \inlinebash{$ ls configs/ | grep am335} to see all predefined
      configurations. The one that supports our board is not obvious:
      it's \code{am335x_evm_defconfig} and not
      \code{am335x_boneblack_vboot_defconfig} which is only for {\em
      verified boot} on BeagleBone Black.
\item So, run \inlinebash{$ make am335x_evm_defconfig}.
}{}

\item Now that you have a valid initial configuration, you can now
  run \inlinebash{$ make menuconfig} to further edit your bootloader features.

Here, though, the default configuration works fine for our needs,
so no change is necessary.

\item Finally, run
  \ifdefstring{\labboard}{beaglebone}{
    \bashcmd{make DEVICE_TREE=am335x-boneblack}
    or \bashcmd{make DEVICE_TREE=am335x-boneblack-wireless}
  }{}
  which will build U-Boot
  \footnote{You can speed up the
  compiling by using the \code{-jX} option with \code{make}, where X
  is the number of parallel jobs used for compiling. Twice the
  number of CPU cores is a good value.}.
  The \code{DEVICE_TREE} variable specifies the specific
  Device Tree that describes our hardware board.
  Alternatively, if you wish to run just \code{make},
  specify our board's device tree name on
  \code{Device Tree Control} $\rightarrow$ \code{Default Device Tree for DT Control}
  option.
\end{enumerate}

\begin{notebox}
Personal workspace: \url{https://github.com/ngocho31/bbb-linux}

Compiling U-Boot run: \inlinebash{$ make bootloader}
\end{notebox}
