\subchapter{Bootloader - U-Boot - Cross-compiling}{Objectives:
Compile the U-Boot bootloader}

As the bootloader is the first piece of software executed by a
hardware platform, the installation procedure of the bootloader is
very specific to the hardware platform. There are usually two cases:

\begin{itemize}

\item The processor offers nothing to ease the installation of the
  bootloader, in which case the JTAG has to be used to initialize
  flash storage and write the bootloader code to flash. Detailed
  knowledge of the hardware is of course required to perform these
  operations.

\item The processor offers a monitor, implemented in ROM, and through
  which access to the memories is made easier.

\end{itemize}

\ifdefstring{\labboard}{beaglebone}
{
The AM3358 SoC on the BeagleBone falls into the second category. The monitor
integrated in the ROM reads the SD card to search for a valid
bootloader.
}{}

\ifdefstring{\labboard}{stm32}
{
The STM32MP1 SoC, falls into the second category. The monitor
integrated in the ROM reads the SD card to search for a valid
bootloader (the boot mode is actually configurable via a few input
pins). In case no bootloader is found, it will operate in a fallback
mode, that will allow to use an external tool to reflash some
executable through USB. Therefore, either by using an MMC/SD card or
that fallback mode, we can start up an STM32MP1-based board without
having anything installed on it.
}{}

\section{Compiling U-Boot and SPL}

Get an understanding of U-Boot's configuration and compilation steps
by reading the \code{README} file, and specifically the {\em Building
the Software} section.

Basically, you need to:

\begin{enumerate}
\item Specify the cross-compiler prefix
(the part before \code{gcc} in the cross-compiler executable name):

\begin{bashinput}
$ export PATH=$PATH:$HOME/x-tools/%\ifdefstring{\arch}{ARM64}{aarch64-none-linux-musl}{arm-none-linux-musleabihf}%/bin/
$ export CROSS_COMPILE=arm-linux-
\end{bashinput}

\ifdefstring{\labboard}{beaglebone}
{
\item Run \inlinebash{$ ls configs/ | grep am335} to see all predefined
      configurations. The one that supports our board is not obvious:
      it's \code{am335x_evm_defconfig} and not
      \code{am335x_boneblack_vboot_defconfig} which is only for {\em
      verified boot} on BeagleBone Black.
\item So, run \inlinebash{$ make am335x_evm_defconfig}.
}{}

\ifdefstring{\labboard}{stm32}
{
\item Run \inlinebash{$ make <NAME>_defconfig}, where the list of available
  configurations can be found in the \code{configs/} directory. There
  are multiple stm32mp15 configurations. We will use the standard one
  (\code{stm32mp15}).
}{}

\item Now that you have a valid initial configuration, you can now
  run \inlinebash{$ make menuconfig} to further edit your bootloader features.

  \ifdefstring{\labboard}{stm32}
  {
  \begin{itemize}

  \item In the \code{Environment} submenu, we will configure U-Boot so
    that it stores its environment inside a file called \code{uboot.env}
    in an ext4 filesystem:
    \begin{itemize}
    \item Disable \code{Environment is not stored}. We want changes to variables to
        be persistent across reboots
    \item Enable \code{Environment is in a EXT4 filesystem}. Disable all other
        options for environment storage (e.g. MMC, SPI, UBI)
% Environment in MMC also works, but we need to be careful with the size available
% and the offset defined. tests showed that if MMC is selected it will take
% precedence over ext4
    \item The value for \code{Name of the block device for the environment} should be \code{mmc}
    \item The value for \code{Device and partition for where to store
      the environment in EXT4} should be \code{0:4}, which indicates
      we want to store the environment in the 4th partition of the
      first MMC device.
    \item The value for \code{Name of the EXT4 file to use for the
      environment} should be \code{/uboot.env}, which indicates the
      filename inside which the U-Boot environment will be stored
    \end{itemize}

  \item In the \code{Device Drivers} $\rightarrow$ \code{Watchdog
      Timer Support} submenu, disable \code{IWDG watchdog
      driver for STM32 MP's family}, so that U-Boot doesn't start the
    watchdog.
  \end{itemize}
  }{}

\item Finally, run
  \ifdefstring{\labboard}{beaglebone}{
    \bashcmd{make DEVICE_TREE=am335x-boneblack}
    or \bashcmd{make DEVICE_TREE=am335x-boneblack-wireless}
  }{}
  \ifdefstring{\labboard}{stm32}
  {
    \bashcmd{make DEVICE_TREE=stm32mp157a-dk1}
  }{}
  which will build U-Boot
  \footnote{You can speed up the
  compiling by using the \code{-jX} option with \code{make}, where X
  is the number of parallel jobs used for compiling. Twice the
  number of CPU cores is a good value.}.
  The \code{DEVICE_TREE} variable specifies the specific
  Device Tree that describes our hardware board.
  Alternatively, if you wish to run just \code{make},
  specify our board's device tree name on
  \code{Device Tree Control} $\rightarrow$ \code{Default Device Tree for DT Control}
  option.
\end{enumerate}

\begin{notebox}
Personal workspace: \url{https://github.com/ngocho31/bbb-linux}

Compiling U-Boot run: \inlinebash{$ make bootloader}
\end{notebox}
