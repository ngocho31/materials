\subchapter{Bootloader - TF-A}{Objectives: Set up TF-A.}

\section{TF-A and U-Boot relationship}

The boot process is done in two steps with the ROM monitor trying to
execute a first piece of software, called {\em fsbl}, from its
internal SRAM, that will initialize the DRAM, and a second program,
{\em ssbl} that will in turn load Linux and execute it.

In our case, {\em fsbl} is provided by TF-A BL2 and {\em ssbl}
is provided by U-Boot.

TF-A BL2 is loading U-Boot from the Firmware Image Package ({\em FIP}),
that will also contain the configuration for this second part.
The {\em FIP} is generated from TF-A sources, so we will build U-Boot
first.

\section{TF-A setup}

Get the mainline TF-A sources:

\begin{bashinput}
$ cd ..
$ git clone https://git.trustedfirmware.org/TF-A/trusted-firmware-a.git
$ cd trusted-firmware-a/
$ git checkout v2.12.0
\end{bashinput}

Several configuration parameters have to be passed to the Makefile:
\begin{itemize}
\item Specify the cross-compiler prefix (the part before gcc in the
      cross-compiler executable name), either using the environment
      variable:\inlinebash{$ export CROSS_COMPILE=arm-linux-}, or just by
      adding it to the \code{make} command line.
\item The architecture has to be selected: \code{ARCH=aarch32}, as
      well as the major version of Arm Architecture, here the Cortex A7 is
      an Armv7, so we need to use \code{ARM_ARCH_MAJOR=7}
\item The STM32MP1 platform is selected too with \code{PLAT=stm32mp1}
\item Specify the AArch32 Secure Payload component, we are going to
      use a minimal monitor implementation provided by TF-A: the
      {\em SP-MIN}. For this we need to add the following variable:
      \code{AARCH32_SP=sp_min}
\item For this specific board, the device tree is generated and then
      needs to be specifed: \code{DTB_FILE_NAME=stm32mp157a-dk1.dtb}
\item Specify the configuration of this firmware which is actually
      the Device Tree passed to U-Boot:
      \code{BL33_CFG=../u-boot/u-boot.dtb}
\item Specify that the fsbl will be located on the SD
      card with \code{STM32MP_SDMMC=1}.
\item Specify the location of the BL33 (Boot loader stage 3-3):
      \code{BL33=../u-boot/u-boot-nodtb.bin}
\end{itemize}

We can now generate the \code{bl32}, \code{dtb}, and \code{fip} targets
with a single command line:
\begin{bashinput}
$ make ARM_ARCH_MAJOR=7 ARCH=aarch32 PLAT=stm32mp1 AARCH32_SP=sp_min \
  DTB_FILE_NAME=stm32mp157a-dk1.dtb BL33=../u-boot/u-boot-nodtb.bin \
  BL33_CFG=../u-boot/u-boot.dtb STM32MP_SDMMC=1 fip all
\end{bashinput}

At the end of the build, the important output files generated are
located in \code{build/stm32mp1/release/}. We will find there:

\begin{itemize}

\item \code{tf-a-stm32mp157a-dk1.stm32}, which is TF-A BL2, serving as
  our first stage bootloader

\item \code{fip.bin}, which is the FIP image, which itself includes
  U-Boot. This image will serve as the second stage bootloader.

\end{itemize}
