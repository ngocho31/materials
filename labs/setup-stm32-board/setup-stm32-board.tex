\subchapter{Setup STM32MP1 board}{Objectives: Set up serial
  communication with the development workstation.}

\section{Setting up serial communication with the board}

Plug the USB-A to micro USB-B cable on the Discovery board. There is
only one micro USB port on the board, it is CN11, also named ST-LINK.
This is a debug interface and exposes multiple debugging interfaces,
including a serial interface. When plugged in your computer, a serial
port should appear, {\tt \hosttty}.

You can also see this device appear by looking at the output of
\code{sudo dmesg}.

To communicate with the board through the serial port, install a
serial communication program, such as \code{picocom}:

\bashcmd{$ sudo apt install picocom}

If you run {\tt ls -l \hosttty}, you can also see that only
\code{root} and users belonging to the \code{dialout} group have
read and write access to the serial console. Therefore, you need
to add your user to the \code{dialout} group:

\bashcmd{$ sudo adduser $USER dialout}

{\bf Important}: for the group change to be effective, you have to
reboot your computer (at least on Ubuntu 20.04) and log in again.
A workaround is to run \code{newgrp dialout}, but it is not global.
You have to run it in each terminal.

Run {\tt picocom -b 115200 \hosttty}, to start serial
communication on {\tt \hosttty}, with a baudrate of 115200.
If you wish to exit \code{picocom}, press \code{[Ctrl][a]} followed by
\code{[Ctrl][x]}.

Don't be surprised if you don't get anything on the serial console yet,
even if you reset the board. That's because the SoC has nothing to boot
on yet. We will prepare a micro SD card to boot on in the next paragraphs.
