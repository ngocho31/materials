\subchapter{Bootloader - U-Boot Installing}{Objectives: Install
  the U-Boot bootloader, use basic U-Boot commands}

\section{Flashing the bootloaders}

The ROM monitor will look for the first stage bootloader in a
partition named \code{fsbl1}. If it cannot find a valid bootloader in
this partition, it will then try to load it from a partition named
\code{fslb2}. This first stage bootloader (in our case the TF-A BL2)
will load the second bootloader (U-Boot) from the Firmware Image
Package located in the partition named \code{fip}. At the same time,
BL2 will also load the BL32 monitor (SP-min) from the FIP. Finally,
U-Boot will store its environment in the fourth partition, which we'll
name \code{bootfs}.

So, as far as bootloaders are concerned, the SD card partitioning will
look like:

\begin{verbatim}
Number  Start   End      Size     File system  Name    Flags
 1      2048s   4095s    2048s                 fsbl1
 2      4096s   6143s    2048s                 fsbl2
 3      6144s   10239s   4096s                 fip
 4      10240s  131071s  120832s               bootfs
\end{verbatim}

On your workstation, plug in the SD card your instructor gave you. Type
the \code{sudo dmesg} command to see which device is used by your
workstation. In case the device is \code{/dev/mmcblk0}, you will see
something like

\begin{verbatim}
[46939.425299] mmc0: new high speed SDHC card at address 0007
[46939.427947] mmcblk0: mmc0:0007 SD16G 14.5 GiB
\end{verbatim}

The device file name may be different (such as \code{/dev/sdb}
if the card reader is connected to a USB bus (either internally
or using a USB card reader).

In the following instructions, we will assume that your SD card is
seen as \code{/dev/mmcblk0} by your PC workstation.

Type the \code{mount} command to check your currently mounted
partitions. If SD partitions are mounted, unmount them:

\bashcmd{$ sudo umount /dev/mmcblk0p*}

We will erase the existing partition table and partition contents
by simply zero-ing the first 128 MiB of the SD card:

\bashcmd{$ sudo dd if=/dev/zero of=/dev/mmcblk0 bs=1M count=128}

Now, let's use the \code{parted} command to create the partitions that
we are going to use:

\bashcmd{$ sudo parted /dev/mmcblk0}

The ROM monitor handles {\em GPT} partition tables, let's create one:

\begin{verbatim}
(parted) mklabel gpt
\end{verbatim}

Then, the 4 partitions are created with:
\begin{verbatim}
(parted) mkpart fsbl1 0% 4095s
(parted) mkpart fsbl2 4096s 6143s
(parted) mkpart fip 6144s 10239s
(parted) mkpart bootfs 10240s 131071s
\end{verbatim}

You can verify everything looks right with:

\begin{verbatim}
(parted) print
Model: SD SA08G (sd/mmc)
Disk /dev/mmcblk0: 7747MB
Sector size (logical/physical): 512B/512B
Partition Table: gpt
Disk Flags:

Number  Start   End     Size    File system  Name    Flags
 1      1049kB  2097kB  1049kB               fsbl1
 2      2097kB  3146kB  1049kB               fsbl2
 3      3146kB  5243kB  2097kB               fip
 4      5243kB  67.1MB  61.9MB               bootfs

(parted)
\end{verbatim}

Once done, quit:
\begin{verbatim}
(parted) quit
\end{verbatim}

{\em Note: \code{parted} is definitely not very user friendly compared
to other tools to manipulate partitions (such as \code{cfdisk}), but
that's the only tool which supports assigning names to GPT partitions.
In your projects, you could use \code{gparted}, which is a more
friendly graphical front-end on top of \code{parted}.}

Now, format the boot partition as an ext4 filesystem. This is where
U-Boot saves its environment:
\bashcmd{$ sudo mkfs.ext4 -L boot -O ^metadata_csum /dev/mmcblk0p4}

The \code{-O ^metadata_csum} option allows to create the filesystem
without enabling metadata checksums, which U-Boot doesn't seem to
support yet.

Now write the TF-A binary in both \code{fsbl} partitions:

\begin{bashinput}
$ sudo dd if=build/stm32mp1/release/tf-a-stm32mp157a-dk1.stm32 of=/dev/mmcblk0p1 bs=1M conv=fdatasync
$ sudo dd if=build/stm32mp1/release/tf-a-stm32mp157a-dk1.stm32 of=/dev/mmcblk0p2 bs=1M conv=fdatasync
\end{bashinput}

Then flash the {\em fip} partition with the Firmware Image Package
containing U-Boot, the BL32 monitor and their configuration (device tree):

\bashcmd{$ sudo dd if=build/stm32mp1/release/fip.bin of=/dev/mmcblk0p3 bs=1M conv=fdatasync}

\section{Testing the bootloaders}

Insert the SD card in the board slot. You can now power-up the board
by connecting the USB-C cable to the board, in CN6, \code{PWR_IN} and
to your PC at the other end. Check that it boots your new bootloaders.
You can verify this by checking the build dates:

\begin{verbatim}
NOTICE:  Model: STMicroelectronics STM32MP157A-DK1 Discovery Board
NOTICE:  Board: MB1272 Var3.0 Rev.C-02
NOTICE:  BL2: v2.12.0(release):v2.12.0
NOTICE:  BL2: Built : 11:38:56, Nov 30 2024
NOTICE:  BL2: Booting BL32
NOTICE:  SP_MIN: v2.12.0(release):v2.12.0
NOTICE:  SP_MIN: Built : 11:38:49, Nov 30 2024


U-Boot 2024.10 (Nov 30 2024 - 11:37:02 +0100)

CPU: STM32MP157DAC Rev.Z
Model: STMicroelectronics STM32MP157A-DK1 Discovery Board
Board: stm32mp1 in trusted mode (st,stm32mp157a-dk1)
Board: MB1272 Var3.0 Rev.C-02
DRAM:  512 MiB
Clocks:
- MPU : 650 MHz
- MCU : 208.878 MHz
- AXI : 266.500 MHz
- PER : 24 MHz
- DDR : 533 MHz
Core:  310 devices, 36 uclasses, devicetree: board
NAND:  0 MiB
MMC:   STM32 SD/MMC: 0
Loading Environment from EXT4... ** File not found /uboot.env **

** Unable to read "/uboot.env" from mmc0:4 **
In:    serial
Out:   serial
Err:   serial
Previous ADC measurements was not the one expected, retry in 20ms
****************************************************
*        WARNING 500mA power supply detected       *
*     Current too low, use a 3A power supply!      *
****************************************************

Net:   eth0: ethernet@5800a000

Hit any key to stop autoboot:  0
STM32MP>
\end{verbatim}

In U-Boot, type the \code{help} command, and explore the few commands
available.

\subsection{Adding a new command to the U-Boot shell}

Check whether the \code{config} command is available. This command
allows to dump the configuration settings U-Boot was compiled from.

If it's not, go back to U-Boot's configuration and enable it.

Re-run the build of U-Boot, and then re-run the build of TF-A so that
a new version of the \code{fip.bin} with the updated U-Boot is
generated.

Update the \code{fip} partition on the SD card with the new
\code{fip.bin} image and test that the command is now available and
works as expected.

\subsection{Playing with the U-Boot environment}

Display the U-Boot environment using \code{printenv}.

Set a new U-Boot variable \code{foo} to a value of your choice, using
\code{setenv}, and verify it has been set. Reset the board, and check
if \code{foo} is still defined: it should not.

Now repeat this process, but before resetting the board, use
\code{saveenv}. After the reset, check the \code{foo} variable is
still defined.

Now reset the environment to its default settings using \code{env
  default -a}, and save these changes using \code{saveenv}.

\section{Setting up networking}

The next step is to configure U-boot and your workstation to let your
board download files, such as the kernel image and Device Tree Binary
(DTB), using the TFTP protocol through a network connection.

With a network cable, connect the Ethernet port of
your board to the one of your computer. If your computer already has a
wired connection to the network, your instructor will provide you with
a USB Ethernet adapter. A new network interface should appear on your
Linux system.

\subsection{Network configuration on the target}

Let's configure networking in U-Boot:

\begin{itemize}
  \item \code{ipaddr}: IP address of the board
  \item \code{serverip}: IP address of the PC host
\end{itemize}

\begin{ubootinput}
=> setenv ipaddr 192.168.0.100
=> setenv serverip 192.168.0.1
\end{ubootinput}

Of course, make sure that this address belongs to a separate network
segment from the one of the main company network.

To make these settings permanent, save the environment:

\begin{ubootinput}
=> saveenv
\end{ubootinput}

\subsection{Network configuration on the PC host}

To configure your network interface on the workstation side, we need
to know the name of the network interface connected to your board.

Find the name of this interface by typing:
\bashcmd{=> ip a}

The network interface name is likely to be
\code{enxxx}\footnote{Following the {\em Predictable Network Interface
Names} convention:
\url{https://www.freedesktop.org/wiki/Software/systemd/PredictableNetworkInterfaceNames/}}.
If you have a pluggable Ethernet device, it's easy to identify as it's
the one that shows up after pluging in the device.

Then, instead of configuring the host IP address from NetworkManager's
graphical interface, let's do it through its command line interface,
which is so much easier to use:

\bashcmd{$ nmcli con add type ethernet ifname en... ip4 192.168.0.1/24}

\section{Setting up the TFTP server}

Let's install a TFTP server on your development workstation:

\begin{verbatim}
sudo apt install tftpd-hpa
\end{verbatim}

You can then test the TFTP connection. First, put a small text file in
the directory exported through TFTP on your development
workstation. Then, from U-Boot, do:

\begin{ubootinput}
=> tftp %\zimageboardaddr% textfile.txt
\end{ubootinput}

In case the download fails, make sure your host interface is correctly
configured and if a firewall is enabled make sure it does not filter out our
requests:
\begin{verbatim}
sudo ufw allow from 192.168.0.100
\end{verbatim}

Otherwise the \code{tftp} command should have downloaded the
\code{textfile.txt} file from your development workstation into
the board's memory at location {\tt \zimageboardaddr}\footnote{
This location is part of the board DRAM. If you want
to check where this value comes from, you can check the SoC
datasheet at
\url{https://www.st.com/resource/en/reference_manual/dm00327659.pdf}.
It's a big document (more than 4,000 pages). In this document, look
for \code{Memory organization} and you will find the SoC memory map.
You will see that the address range for the memory controller
({\em DDRC})
starts at the address we are looking for.
You can also try with other values in the RAM address range.}.

You can verify that the download was successful by dumping the
contents of the memory:

\begin{ubootinput}
=> md %\zimageboardaddr%
\end{ubootinput}

We will see in the next labs how to use U-Boot to download, flash and
boot a kernel.

% \input{../common/dk1-known-issues.tex}
