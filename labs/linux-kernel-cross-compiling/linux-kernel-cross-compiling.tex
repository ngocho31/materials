\subchapter{Kernel - Cross-compiling}{Objective: Compile
  the Linux kernel for an ARM target platform.}

\section{Cross-compiling environment setup}

To cross-compile Linux, you need to have a cross-compiling
toolchain. We will use the cross-compiling toolchain that we
previously produced, so we just need to make it available in the PATH:

\begin{bashinput}
$ export PATH=$PATH:$HOME/x-tools/%\ifdefstring{\arch}{ARM64}{aarch64-none-linux-musl}{arm-none-linux-musleabihf}%/bin/
\end{bashinput}

Also, don't forget to either:

\begin{itemize}
\item Define the value of the \code{ARCH} and \code{CROSS_COMPILE}
  variables in your environment (using \code{export})
\item {\bf Or} specify them on the command line at every invocation of
  \code{make}, i.e.: \code{make ARCH=... CROSS_COMPILE=... <target>}
\end{itemize}

% \ifdefstring{\labboard}{beagleplay}{
% \textbf{Important:} The majority of tools use \code{aarch64} to define 64 bits
% ARM processors. However in specific cases, the tools may use \code{arm64} to
% define such architecture. In the case of the kernel Makefile we will use the
% \code{arm64} one.
% }{}

\section{Linux kernel configuration}

By running \code{make help}, look for the proper Makefile target to
configure the kernel for your processor.

% \ifdefstring{\labboard}{stm32mp1}
% {The standard configuration for this kernel is actually \code{multi_v7_defconfig},
% but this will generate a pretty big kernel with support for many other
% SoCs. However, we can reduce it to compile faster and get a small
% kernel.}{}

\ifdefstring{\labboard}{beaglebone}
{In our case look for a configuration for boards in the OMAP2 and
later family which the AM335x found in the BeagleBone belongs to.}{}

% \ifdefstring{\labboard}{beagleplay}
% {If you search for a configuration file that corresponds to the \code{arm64}
% architecture that the kernel will use, you might see that only one
% \code{defconfig} file is available. We will therefore load this basic
% configuration file and modify it later.}{}

% \ifdefstring{\labboard}{qemu}
% {In course case, use the configuration for the ARM Vexpress boards
% (\code{vexpress_defconfig}).}{}

So, apply this configuration, and then run \code{make menuconfig}.

\begin{itemize}
\item Disable \kconfig{CONFIG_GCC_PLUGINS} if it is set. This will skip
  building special {\em gcc} plugins, which would require extra dependencies
  for the build.

% \ifdefstring{\labboard}{stm32mp1}{
%   \item In the \code{System Type} menu, remove support for all the SoCs except
%   the STM32MP157 ones. Don't forget to disable the TI ones too which are
%   in a submenu.
%   \item Disable \kconfig{CONFIG_DRM}, which will skip support for many display
%   controller and GPU drivers.}{}

% \ifdefstring{\labboard}{qemu}{
%   Also start \code{make menuconfig} to
%   add \kconfig{CONFIG_DEVTMPFS_MOUNT} to your configuration.}{}

\ifdefstring{\labboard}{beaglebone}{
\item Add options to support USB host and networking over USB device:
\begin{itemize}
  \item \kconfigval{CONFIG_USB}{y}
  \item \kconfigval{CONFIG_USB_GADGET}{y}
  \item \kconfigval{CONFIG_USB_MUSB_HDRC}{y} {\em Driver for the USB OTG
        controller}
  \item \kconfigval{CONFIG_USB_MUSB_DSPS}{y}
  \item \kconfigval{CONFIG_USB_MUSB_DUAL_ROLE}{y} {\em Use the USB OTG
        both in host and device (gadget) modes. This will be needed
        later to use the board's USB host port.}
  \item Check the dependencies of \kconfig{CONFIG_AM335X_PHY_USB}
        and find the way to set \kconfigval{CONFIG_AM335X_PHY_USB}{y}
  \item Find the {\em USB Gadget precomposed configurations} menu
        and set it to {\em static} instead of {\em module}
        so that \kconfigval{CONFIG_USB_ETH}{y}
\end{itemize}
\item Also compile your kernel with \kconfigval{CONFIG_INPUT_EVDEV}{y},
  to have the same default setting as in our labs with the STM32MP1 boards.
}{}

% \ifdefstring{\labboard}{beagleplay}{
% \item In the \code{Platform Selection} menu, remove support for all the SoCs except
% for the Texas Instruments Inc. K3 multicore SoC architecture.
% \item Disable \kconfig{CONFIG_DRM}, which will skip support for many display
% controller and GPU drivers.
% }{}

\end{itemize}

% \ifdefstring{\labboard}{stm32mp1}{
% Please note that this will definitely not build the smallest
% and most optimized kernel for STM32MP1: \code{multi_v7_defconfig}
% enables plenty of features and drivers that will not be useful on our
% particular board.}{}

% \ifdefstring{\labboard}{beagleplay}{
% Please note that this will definitely not build the smallest and most
% optimized kernel for the BeaglePlay: the ARM64 \code{defconfig}
% enables plenty of features and drivers that will not be useful on our
% particular board.}{}

\section{Cross compiling}

You're now ready to cross-compile your kernel. Simply run:

\bashcmd{$ make}

and wait a while for the kernel to compile. Don't forget to use
\code{make -j<n>} if you have multiple cores on your machine!

Look at the kernel build output to see which file contains
the kernel image.

Also look in the Device Tree Source directory to see which \code{.dtb}
files got compiled. Find which \code{.dtb} file corresponds to your board.

\ifdefstring{\labboard}{beaglebone}
{For the BeagleBone Black, its \code{am335x-boneblack.dtb},
and for the BeagleBone Black Wireless, its
\code{am335x-boneblack-wireless.dtb}.}
{}

\begin{notebox}
Personal workspace: \url{https://github.com/ngocho31/bbb-linux}

Compiling kernel run: \inlinebash{$ make kernel}
\end{notebox}
