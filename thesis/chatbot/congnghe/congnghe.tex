\chapter{Công nghệ sử dụng}

\section{Ngôn ngữ lập trình}

\subsection{Python}
Python là một ngôn ngữ lập trình bậc cao cho các mục đích lập trình
đa năng, do Guido van Rossum tạo ra và lần đầu ra mắt vào năm 1991
và được phát triển trong một dự án mã nguồn mở. Python sử dụng
cơ chế cấp phát bộ nhớ động, hỗ trợ các phương thức lập trình như
lập trình hướng đối tượng, lập trình hàm.

Python được thiết kế với ưu điểm mạnh là dễ đọc, dễ học và dễ nhớ.
Python là ngôn ngữ có hình thức rất sáng sủa, cấu trúc rõ ràng,
thuận tiện cho người mới học lập trình và là ngôn ngữ lập trình
dễ học, được dùng rộng rãi trong phát triển trí tuệ nhân tạo.

Python hỗ trợ rất mạnh về Machine Learning, Deep Learning với các
thư viện như TensorFlow, Pytorch, Keras, ... Ngoài ra còn có các
thư viện hỗ trợ mạnh cho việc xử lý dữ liệu như là Pandas, Numpy,
Scipy, ...

Trong đề tài này sử dụng phiên bản Python 3.8 để hiện thực các
thành phần trong hệ thống.

\subsection{JavaScript}
JavaScript được tạo trong mười ngày bởi Brandan Eich, một nhân viên
của Netscape, vào tháng 9 năm 1995. JavaScript liên tục phát triển
kể từ đó. Chỉ sau 20 năm, nó từ một ngôn ngữ lập trình riêng trở thành
công cụ quan trọng nhất trên bộ công cụ của các chuyên viên lập trình web.

Một số đặc điểm của JavaScript:

\begin{itemize}
    \item Là một ngôn ngữ lập trình thông dịch.
    \item Không cần một compiler vì web browser có thể biên dịch nó
    bằng HTML.
    \item Dễ học và dễ sử dụng hơn các ngôn ngữ lập trình khác.
    \item Hoạt động trên nhiều trình duyệt, nền tảng.
\end{itemize}

JavaScript được dùng rộng rãi cho các trang web (phía người dùng)
cũng như phía máy chủ, cũng là một lý do khiến nó trở nên rất phổ biến.

\subsection{HTML}
HTML được tạo ra bởi Tim Berners-Lee, một nhà vật lý học của
trung tâm nghiên cứu CERN ở Thụy Sĩ. Hiện nay, HTML đã trở thành một
chuẩn Internet được tổ chức W3C (World Wide Web Consortium) vận hành
và phát triển.

HTML viết tắt của Hypertext Markup Language, tạm dịch là ngôn ngữ
đánh dấu siêu văn bản, là ngôn ngữ lập trình dùng để xây dựng và
cấu trúc lại các thành phần có trong Website. Nó có thể được trợ giúp
bởi các công nghệ như CSS và các ngôn ngữ kịch bản giống như JavaScript.

Một số đặc điểm của HTML:

\begin{itemize}
    \item Đây là một ngôn ngữ đơn giản rất dễ học và dễ sử dụng.
    \item Rất dễ dàng để trình bày hiệu quả với HTML vì nó có nhiều
    thẻ định dạng.
    \item Đây là một ngôn ngữ đánh dấu vì vậy có thể sử dụng nó một cách
    linh hoạt để thiết kế trang web cùng với văn bản.
    \item Hoạt động trên nhiều trình duyệt, nền tảng.
\end{itemize}

\subsection{CSS}
CSS là chữ viết tắt của Cascading Style Sheets, nó là một ngôn ngữ
được sử dụng để tìm và định dạng lại các phần tử được tạo ra bởi các
ngôn ngữ đánh dấu (HTML). Nói ngắn gọn hơn là ngôn ngữ tạo phong cách
cho trang web. Nếu HTML đóng vai trò định dạng các phần tử trên
website như việc tạo ra các đoạn văn bản, các tiêu đề, bảng, ... thì
CSS sẽ giúp chúng ta có thể thêm style vào các phần tử HTML đó như
đổi bố cục, màu sắc trang, đổi màu chữ, font chữ, thay đổi cấu trúc, ...

CSS được phát triển bởi W3C (World Wide Web Consortium) vào năm 1996,
vì HTML không được thiết kế để gắn tag để giúp định dạng trang web.

Mối tương quan giữa HTML và CSS rất mật thiết. HTML là ngôn ngữ markup
(nền tảng của site) và CSS định hình phong cách (tất cả những gì tạo
nên giao diện website), chúng là không thể tách rời.

\subsection{JSON}
JSON là chữ viết tắt của Javascript Object Notation, đây là một dạng
dữ liệu tuân theo một quy luật nhất định mà hầu hết các ngôn ngữ
lập trình hiện nay đều có thể đọc được.

JSON sử dụng cú pháp Javascript, nhưng định dạng JSON chỉ văn bản như
XML. Ta có thể sử dụng lưu nó vào một tệp (file), một bản ghi (record)
trong cơ sở dữ liệu rất dễ dàng.

JSON có định dạng đơn giản, là một dạng trao đổi dữ liệu trọng lượng
nhẹ (lightweight data-interchange format), xử lý nhanh, dễ hiểu,
dễ dàng sử dụng hơn XML rất nhiều nên tính ứng dụng của nó hiện nay
rất là phổ biến, trong tương lai tới, các ứng dụng sẽ sử dụng JSON
là đa số.

\section{Nền tảng và thư viện}

\subsection{Pandas}
Pandas là một thư viện phần mềm được viết cho ngôn ngữ lập trình
Python để thao tác và phân tích dữ liệu. Với Pandas, người dùng
có thể dễ dàng tạo ra các bảng dữ liệu (Dataframe) và thực hiện
các phép truy vấn, thống kê trên nó, cho phép đọc/ghi các
định dạng file một cách dễ dàng.

\subsection{TensorFlow}
TensorFlow là một thư viện phần mềm mã nguồn mở dành cho máy học
trong nhiều loại hình tác vụ nhận thức và hiểu ngôn ngữ.

TensorFlow nguyên thủy được phát triển bởi đội Google Brain cho
mục đích nghiên cứu và sản xuất của Google và sau đó được phát hành
theo giấy phép mã nguồn mở Apache 2.0.

TensorFlow tích hợp sẵn rất nhiều các thư viện Machine Learning,
Deep Learning, có khả năng tương thích và mở rộng tốt. Trong đề tài
này, sử dụng nó trong việc xây dựng mạng Deep Q-Learning.

\subsection{Keras}
Keras là một thư viện mạng nơ ron được viết bằng Python năm 2015 bởi
một kỹ sư Deep Learning của Google. Ta có thể kết hợp Keras với các
thư viện Deep Learning. Keras được phát triển với trọng tâm là
cho phép thử nghiệm nhanh, việc thử nghiệm nhanh đôi khi sẽ mang lại
kết quả nghiên cứu tốt. Trong đề tài này, sử dụng Keras trong việc
xây dựng mô hình Deep Q-Learning.

\subsection{Eel}
Eel là một thư viện Python nhỏ để tạo các ứng dụng GUI ngoại tuyến
HTML/JS đơn giản, với toàn quyền truy cập vào các khả năng và
thư viện Python. Eel lưu trữ một máy chủ web cục bộ, sau đó cho phép
chú thích các hàm bằng Python để chúng có thể được gọi từ Javascript
và ngược lại. Eel được thiết kế để giảm bớt rắc rối khi viết các
ứng dụng GUI ngắn và đơn giản.

\section{Công cụ}

\subsection{Google Colab}
Colaboratory hay còn gọi là Google Colab, là một sản phẩm từ Google
Research, nó cho phép chạy các dòng code python thông qua
trình duyệt, đặc biệt phù hợp với Data analysis, machine learning
và giáo dục. Colab không cần yêu cầu cài đặt hay cấu hình máy tính,
mọi thứ có thể chạy thông qua trình duyệt, có thể sử dụng tài nguyên
máy tính từ CPU tốc độ cao và cả GPUs và cả TPUs đều được cung cấp.
