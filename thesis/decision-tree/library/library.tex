\chapter{THƯ VIỆN SCIKIT-LEARN}

\section{Sự hình thành}
Scikit-learn ban đầu được đề xuất bởi David Cournapeau trong
một dự án mùa hè của Google vào năm 2007.

Later Matthieu Brucher tham gia dự án trên và bắt đầu sử dụng
nó làm một phần luận văn tiến sĩ của ông ấy.
Vào năm 2010, INRIA bắt đầu tài trợ và phiên bản đầu tiên được
xuất bản (v0.1 beta) vào cuối tháng 1 năm 2010.

Dự án vẫn đang được nghiên cứu bởi một đội ngũ hơn 30 nhà nghiên cứu
đến từ các công ty lớn INRIA, Google, Tinyclues và Python Software Foundation.

\section{Scikit-learn là gì?}
Scikit-learn (Sklearn) là thư viện mạnh mẽ nhất dành cho các thuật toán
học máy được viết trên ngôn ngữ Python. Thư viện cung cấp một tập các công cụ
xử lý các bài toán machine learning và statistical modeling gồm:
classification, regression, clustering, và dimensionality reduction.

Thư viện được cấp phép bản quyền chuẩn FreeBSD và chạy được trên
nhiều nền tảng Linux. Scikit-learn được sử dụng như một tài liệu để học tập.

Để cài đặt scikit-learn trước tiên phải cài thư viện SciPy (Scientific Python).
Những thành phần gồm:

\begin{itemize}
    \item \textbf{Numpy:} Gói thư viện xử lý dãy số và ma trận nhiều chiều.
    \item \textbf{SciPy:} Gói các hàm tính toán logic khoa học.
    \item \textbf{Matplotlib:} Biểu diễn dữ liệu dưới dạng đồ thị 2 chiều, 3 chiều.
    \item \textbf{IPython:} Notebook dùng để tương tác trực quan với Python.
    \item \textbf{SymPy:} Gói thư viện các kí tự toán học.
    \item \textbf{Pandas:} Xử lý, phân tích dữ liệu dưới dạng bảng.
\end{itemize}

Những thư viện mở rộng của SciPy thường được đặt tên dạng SciKits.
Như thư viện này là gói các lớp, hàm sử dụng trong thuật toán học máy
thì được đặt tên là scikit-learn.

Scikit-learn hỗ trợ mạnh mẽ trong việc xây dựng các sản phẩm.
Nghĩa là thư viện này tập trung sâu trong việc xây dựng các yếu tố:
dễ sử dụng, dễ code, dễ tham khảo, dễ làm việc, hiệu quả cao.

Mặc dù được viết cho Python nhưng thực ra các thư viện nền tảng
của scikit-learn lại được viết dưới các thư viện của C để tăng
hiệu suất làm việc.
Ví dụ như: Numpy (Tính toán ma trận), LAPACK, LibSVM và Cython.
