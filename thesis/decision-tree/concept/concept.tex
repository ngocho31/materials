\chapter{KHÁI NIỆM CÂY QUYẾT ĐỊNH (DECISION TREE)}

Decision Tree là là một trong những thuật toán học có giám sát
(supervised learning), được sử dụng thường xuyên và rộng rãi nhất
có thể được áp dụng vào cả hai bài toán phân loại (classification)
và hồi quy (regression).

Việc xây dựng một Decision Tree trên dữ liệu huấn luyện cho trước
là việc đi xác định các câu hỏi và thứ tự của chúng. Một điểm đáng
lưu ý của Decision Tree là nó có thể làm việc với các đặc trưng
(trong các tài liệu về Decision Tree, các đặc trưng thường được gọi là
thuộc tính – attribute) dạng có phân loại (categorical), thường là rời rạc
và không có thứ tự.

Ví dụ, mưa, nắng hay xanh, đỏ, v.v.

Decision Tree cũng làm việc với dữ liệu có vectơ đặc trưng bao gồm
cả thuộc tính dạng có phân loại (categorical) và liên tục (numeric).

Một điểm đáng lưu ý nữa là Decision Tree ít yêu cầu việc chuẩn hoá dữ liệu.
