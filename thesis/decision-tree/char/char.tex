\chapter{ƯU NHƯỢC ĐIỂM CỦA CÂY QUYẾT ĐỊNH (DECISION TREE)}

\section{Một số ưu điểm của Decision Tree}
Cây quyết định là một thuật toán đơn giản và phổ biến.
Thuật toán này được sử dụng rộng rãi bới những lợi ích của nó:

\begin{itemize}
    \item Mô hình sinh ra các quy tắc dễ hiểu cho người đọc,
    tạo ra bộ luật với mỗi nhánh lá là một luật của cây.
    \item Việc chuẩn bị dữ liệu cho một Decision Tree là cơ bản
    hoặc không cần thiết. Các kỹ thuật khác thường đòi hỏi chuẩn hóa dữ liệu,
    cần tạo các biến phụ (dummy variable) và loại bỏ các giá trị rỗng.
    \item Cây quyết định có thể xử lý cả dữ liệu có giá trị bằng số và
    dữ liệu có giá trị là tên thể loại. Tuy nhiên, hiện tại việc triển khai
    trong scikit-learn không hỗ trợ các biến phân loại. Các kỹ thuật khác
    thường chuyên để phân tích các bộ dữ liệu chỉ gồm một loại biến.
    Chẳng hạn, các luật quan hệ chỉ có thể dùng cho các biến tên,
    trong khi mạng nơ-ron chỉ có thể dùng cho các biến có giá trị bằng số.
    \item Cây quyết định là một mô hình hộp trắng. Nếu có thể quan sát
    một tình huống cho trước trong một mô hình, thì có thể dễ dàng giải thích
    điều kiện đó bằng logic Boolean. Mạng nơ-ron là một ví dụ về mô hình hộp đen,
    do lời giải thích cho kết quả quá phức tạp để có thể hiểu được.
    \item Có thể thẩm định một mô hình bằng các kiểm tra thống kê.
    Điều này làm cho ta có thể tin tưởng vào mô hình.
    \item Cây quyết định có thể xử lý tốt một lượng dữ liệu lớn trong
    thời gian ngắn. Có thể dùng máy tính cá nhân để phân tích các lượng
    dữ liệu lớn trong một thời gian đủ ngắn để cho phép các nhà chiến lược
    đưa ra quyết định dựa trên phân tích của Decision Tree.
    \item Chi phí sử dụng cây (tức là dữ liệu dự đoán) được tính theo lôgarit
    trong số điểm dữ liệu được sử dụng để đào tạo cây.
    \item Có khả năng xử lý các vấn đề đa đầu ra.
\end{itemize}

\section{Một số nhược điểm của Decision Tree}
Kèm với đó, Decision Tree cũng có những nhược điểm cụ thể:

\begin{itemize}
    \item Mô hình Decision Tree phụ thuộc rất lớn vào dữ liệu
    của bạn. Thậm chí, với một sự thay đổi nhỏ trong bộ dữ liệu,
    cấu trúc mô hình Decision Tree có thể thay đổi hoàn toàn.
    \item Decision Tree hay gặp vấn đề overfitting.
    \item Khó giải quyết được những vấn đề có dữ liệu phụ thuộc thời gian liên tục.
    \item Dễ xảy ra lỗi khi có quá nhiều lớp chi phí tính toán để
    xây dựng mô hình Decision Tree cao.
\end{itemize}
