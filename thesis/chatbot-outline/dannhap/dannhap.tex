\chapter {Giới thiệu}

\section{Giới thiệu đề tài}
Khách hàng là nguồn sống của bất cứ cửa hàng, doanh nghiệp nào.
Chính vì thế, chăm sóc khách hàng (CSKH) trở thành một trong những
yếu tố sống còn và đòi hỏi rất nhiều đầu tư về công sức và tiền bạc.

Có nhiều hình thức khác nhau để chăm sóc khách hàng như qua email,
qua điện thoại, qua các diễn đàn trực tuyến, và qua tin nhắn trực tuyến.

Ở nước ta, việc giải đáp thắc mắc của khách hàng qua tin nhắn trực tuyến
đang được ưa chuộng. Tuy nhiên, việc này còn thực hiện một cách thủ công
và gặp nhiều khó khăn như: tốn rất nhiều thời gian và chi phí chi trả
cho nhân viên chỉ để trả lời những câu hỏi đơn giản và giống nhau.

Chính vì vậy, nhu cầu cấp thiết là cần một hệ thống điều khiển thông minh,
tự động để mang lại hiệu quả cao hơn và Chatbot là một sự lựa chọn hoàn hảo.

Cụ thể, tác dụng của Chatbot trong lĩnh vực chăm sóc khách hàng như sau:

\begin{itemize}
    \item Đưa thông tin chính xác tới từng tệp khách hàng.
    \item Trả lời tự động mọi câu hỏi của khách hàng đưa ra mọi lúc mọi nơi.
    \item Tăng sự tương tác của khách hàng và doanh nghiệp.
    \item Tự động chăm sóc khách hàng thường xuyên 24/7.
    \item Giảm chi phí đầu tư.
\end{itemize}

Chatbot chăm sóc khách hàng phù hợp với nhiều loại mô hình doanh nghiệp
từ kinh doanh online (cung cấp thông tin sản phẩm, đưa ra các thông tin
gợi ý...), hay là các nhà hàng, rạp chiếu phim (cung cấp các tùy chọn
menu, chọn vị trí chổ ngồi, thanh toán...), và cũng được sử dụng nhiều
trong lĩnh vực y tế, chăm sóc sức khoẻ.

\section{Mục tiêu của đề tài}
\label{sec:muctieu}
Trong phạm vi nghiên cứu của đề tài này, tập trung xây dựng một hệ thống
hội thoại tự động có thể tư vấn cho khách hàng thông tin về các sản phẩm
thời trang như quần áo, váy đầm, v.v.

Cụ thể, hệ thống bao gồm các tính năng sau:

\begin{itemize}
    \item Tìm kiếm và thu thập dữ liệu phù hợp với nội dung đề tài.
    Lọc nhiễu, trích xuất các thông tin cần thiết để lưu trữ vào cơ sở
    dữ liệu phục vụ cho nhu cầu truy vấn cho Chatbot.
    \item Xây dựng một công cụ đầy đủ các chức năng để quản lý cở sở
    dữ liệu, cung cấp nguồn dữ liệu tin cậy để Chatbot có thể tư vấn
    đầy đủ và chính xác cho người dùng.
    \item Hiểu được ý định, nhu cầu của người dùng khi họ tham gia
    hội thoại với Chatbot, trích xuất được các thông tin mà người dùng
    cung cấp để truy vấn chính xác, thoả mãn yêu cầu của người dùng.
    \item Giao tiếp với người dùng một cách linh hoạt, bám sát với
    luồng hội thoại để mang lại trải nghiệm tốt nhất có thể cho người dùng.
\end{itemize}

Để có thể đạt được những tính năng trên, cần xác định một số công việc
phải giải quyết như sau:

\begin{itemize}
    \item Tích hợp được bộ thu thập dữ liệu, bộ trích xuất thông tin
    để lưu vào cơ sở dữ liệu.
    \item Khảo sát nhu cầu của người dùng khi cần được tư vấn thông tin
    sản phẩm, từ đó xây dựng bộ nhận diện ý định người dùng.
    \item Huấn luyện các mô hình học máy và tích hợp các hệ thống
    đi kèm để có thể đưa ra quyết định cho mỗi hành động khi giao tiếp
    với người dùng cũng như truy vấn dữ liệu chính xác để trả về
    đúng thông tin người dùng mong muốn.
    \item Tích hợp bộ sinh câu phản hồi của Chatbot với ngôn từ
    tự nhiên tạo cảm giác thoải mái cho người dùng.
    \item Tích hợp với bộ giao diện ứng dụng thân thiện, dễ sử dụng.
\end{itemize}

\section{Giới hạn của đề tài}
Các nhu cầu của khách hàng trong lĩnh vực tư vấn thời trang là
rất phong phú nên việc đáp ứng hết tất cả nhu cầu là rất khó khăn.
Vì vậy, trong đề tài này tôi sẽ cố gắng đáp ứng được tất cả nhu cầu
đã được định nghĩa.

Trong đề tài này, chỉ tập trung vào việc nghiên cứu và sử dụng
mô hình học tăng cường huấn luyện cho Chatbot để mang lại độ
chính xác lẫn tự nhiên có thể chấp nhận được, mang lại
trải nghiệm thoải mái cho người dùng.

Đồng thời, xây dựng bộ mô phỏng người dùng và tạo lỗi để tự động
sinh ra các mẫu hội thoại có tính tự nhiên và dùng nó để
huấn luyện cho Chatbot.

Kết quả cuối cùng của đề tài là ứng dụng Chatbot với giao diện
có thể giao tiếp với người dùng dễ dàng nên trong đề tài này,
sẽ tích hợp với các bộ phận khác của một mô hình Chatbot có sẵn.

\section{Các công trình liên quan}
\begin{itemize}
    \item \textbf{SuperAgent: A Customer Service Chatbot for E-commerce Websites - \cite{superagent}}
    \begin{itemize}
        \item Trong bài báo này, họ giới thiệu SuperAgent, một chatbot dịch vụ khách hàng,
        sử dụng dữ liệu thương mại điện tử quy mô lớn và công khai.
        SuperAgent tận dụng dữ liệu từ mô tả sản phẩm trong trang cũng như
        nội dung do người dùng tạo từ các trang web thương mại điện tử.
        \item Ngoài ra, SuperAgent sinh câu phản hồi dựa trên bốn mô hình chạy song song,
        bao gồm các bộ câu hỏi và trả lời (QA) thực tế, bộ tìm kiếm QA thường gặp,
        bộ QA văn bản định hướng ý kiến, và mô hình cuộc trò chuyện chit-chat.
    \end{itemize}
    \item \textbf{Automated Medical Chatbot - \cite{automatedmedical}}
    \begin{itemize}
        \item Trong bài báo này, tác giả đã sử dụng phương pháp AIML
        (Artificial Intelligence Mark-up Language).
        \item Hệ thống sẽ thu thập các thông tin từ người dùng như triệu chứng,
        sau đó đưa ra những căn bệnh mà người dùng có thể mắc phải và
        hỏi người dùng về cảm giác của họ. Sau khi nhận được nhiều dữ liệu,
        nó sẽ tìm ra căn bệnh có khả năng xảy ra nhất.
        \item Tuỳ vào mức độ nghiêm trọng của bệnh mà nó sẽ đề xuất các biện pháp
        khắc phục và thuốc cho người dùng hoặc kết nối người dùng với bác sĩ.
    \end{itemize}
    \item \textbf{Training a Goal-Oriented Chatbot with Deep Reinforcement Learning
    - \cite{traininggochatbot}}
    \begin{itemize}
        \item Đây là chuỗi bài hướng dẫn được thực hiện bởi
        một kênh nổi tiếng trên diễn đàn Medium - Towards Data Science.
        \item Điểm nổi bật của chuỗi bài này là tác giả đã xây dựng được
        một kiến trúc hệ thống học tăng cường (Reinforcement Learning)
        khá hoàn thiện, áp dụng cho bài toán tư vấn có mục tiêu cụ thể.
        \item Đặc biệt, có sử dụng hệ thống giả lập người dùng (user simulator)
        với một số luật đơn giản để giúp hệ thống học tăng cường có thể
        học được nhanh hơn rất nhiều thay vì phải cần người thật tương tác.
        \item Những kiến thức đó được tác giả vận dụng từ một bài báo
        \cite{endtoend} của nhóm nghiên cứu đến từ phòng nghiên cứu
        của Microsoft, Hoa Kỳ.
        \item Cụ thể, kiến trúc mà tác giả đề cập tới bao gồm bốn phần chính là
        \textit{agent} (đối tượng sẽ được huấn luyện để ra quyết định),
        \textit{dialog state tracker} (đối tượng quản lý trạng thái hội thoại),
        \textit{user simulator} (đối tượng giả lập người dùng -
        mục đích giúp quá trình huấn luyện nhanh chóng hơn) và
        \textit{EMC - Error Model Controller} (mô phỏng lỗi -
        giúp \textit{agent} học hiệu quả hơn).
    \end{itemize}
\end{itemize}

\section{Cấu trúc đề cương}
Trong giai đoạn đề cương đề tài đã thực hiện được một số công việc
liên quan sẽ trình bày trong báo cáo như sau:

\begin{itemize}
\item Chương 1: Giới thiệu tổng quan về đề tài, mục tiêu và giới hạn
sẽ thực hiện trong đề tài. Đồng thời trình bày một số công trình
có liên quan tới đề tài.
\item Chương 2: Trình bày các kiến thức nền tảng có liên quan tới đề tài.
\item Chương 3: Trình bày hướng tiếp cận bài toán: giới thiệu về
kiến trúc hệ thống, chi tiết cách tiếp cận thực hiện mô hình.
\item Chương 4: Trình bày kế hoạch triển khai và các giai đoạn sẽ thực hiện.
\end{itemize}
